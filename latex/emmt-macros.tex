%% To avoid errors like "Command \iint already defined.", use AMS math packages
%% *before* other packages such as `txfonts`.
\usepackage{amsmath,amsfonts,amssymb}
\usepackage{mathtools}

%%%%%%%%%%%%%%%%%%%%%%%%%%%%%%%%%%%%%%%%%%%%%%%%%%%%%%%%%%%%%%%%%%%%%%%%%%%%%%%
%% Page setup.
%\usepackage[left=9mm, right=9mm, top=20mm, bottom=20mm,columnsep=8mm]{geometry}

%%%%%%%%%%%%%%%%%%%%%%%%%%%%%%%%%%%%%%%%%%%%%%%%%%%%%%%%%%%%%%%%%%%%%%%%%%%%%%%
%% Graphics, colors and hyperlinks.

\usepackage{graphicx}
\graphicspath{{figs/}}

%\usepackage[dvipsnames,usenames]{color}
%\usepackage[dvipsnames,hyperref]{xcolor}
\newcommand{\oops}[1]{\textcolor[named]{BrickRed}{#1}}

%\colorlet{HyperColor}{MidnightBlue}
%\colorlet{ChapColor}{OliveGreen}
%\colorlet{MyLinkColor}{HyperColor}
%\colorlet{MyURLColor}{HyperColor}
%\colorlet{MyCiteColor}{HyperColor}
%\colorlet{MyFileColor}{HyperColor}
%\colorlet{MyMenuColor}{HyperColor}
%\colorlet{MyPageColor}{HyperColor}
%\hypersetup{
%  %pdftitle={\TheTitle},
%  %pdfauthor={\TheAuthors},
%  unicode=true,
%  linkcolor=MyLinkColor,
%  urlcolor=MyURLColor,
%  citecolor=MyCiteColor,
%  filecolor=MyFileColor,
%  menucolor=MyMenuColor}

%\usepackage[unicode=true,
% bookmarks=true,bookmarksnumbered=false,bookmarksopen=false,
% breaklinks=true,pdfborder={0 0 0},backref=page,colorlinks=true,
% citecolor=MidnightBlue,]{hyperref}

%%%%%%%%%%%%%%%%%%%%%%%%%%%%%%%%%%%%%%%%%%%%%%%%%%%%%%%%%%%%%%%%%%%%%%%%%%%%%%%
%% Fonts.

%\usepackage{txfonts}
%\usepackage{lmodern}

% See https://tex.stackexchange.com/questions/204998/double-struck-zero-and-one
% and https://tex.stackexchange.com/questions/215822/use-both-boondox-ds-and-the-blackboard-fonts-from-amsfonts
% for hints to have double-struck numbers.
\DeclareMathAlphabet{\mymathbb}{U}{BOONDOX-ds}{m}{n}

\usepackage[bb=ams]{mathalfa}
%\usepackage{bbm}
%\usepackage{dsfont}
%\usepackage[nointegrals]{wasysym} % avoid clash with amsmath \iint and \iiint

%%%%%%%%%%%%%%%%%%%%%%%%%%%%%%%%%%%%%%%%%%%%%%%%%%%%%%%%%%%%%%%%%%%%%%%%%%%%%%%
%% Algorithms.

\usepackage[ruled,vlined]{algorithm2e}
\DontPrintSemicolon
\SetKw{KwAnd}{and}
\SetKw{KwOr}{or}
\SetKw{KwBreak}{break}

%\usepackage{algpseudocode}

%\renewcommand{\CommentSty}[1]{\small #1}
%\SetKwComment{Comment}{}{}
\SetKwComment{Comment}{\ensuremath{\vartriangleleft} }{}
%\newcommand{\CommentFormat}[1]{\textbf{\small #1}}
\newcommand{\CommentFormat}[1]{{\small #1}}
\SetCommentSty{CommentFormat}
\newcommand{\CommentLine}[1]{{\normalsize\bf\ensuremath{\blacksquare} #1}\;}

% For side-comments (normally produced by \Comment*{TEXT}), when the text
% is too long, \SideCommentL{WIDTH}{TEXT} or \SideCommentL{WIDTH}{TEXT}
% with produce a multi-line comment of given width and following style:
%   - \SideCommentL: left-justified and vertically centered;
%   - \SideCommentR: right-justified and top line vertically aligned;
\newcommand{\SideCommentL}[2]{\Comment*{\parbox[t]{#1}{\raggedleft%
  \CommentSty{#2}}}}
\newcommand{\SideCommentR}[2]{\Comment*{\parbox{#1}{\raggedright%
  \CommentSty{#2}}}}

%%%%%%%%%%%%%%%%%%%%%%%%%%%%%%%%%%%%%%%%%%%%%%%%%%%%%%%%%%%%%%%%%%%%%%%%%%%%%%%
%% Drawing.

%\usepackage{tikz}
%\usetikzlibrary{calc}
%\usetikzlibrary{arrows.meta}

%%%%%%%%%%%%%%%%%%%%%%%%%%%%%%%%%%%%%%%%%%%%%%%%%%%%%%%%%%%%%%%%%%%%%%%%%%%%%%%
%% Tables.

% Table settings:
%\usepackage{array}
%\usepackage{booktabs}
%\colorlet{tableheadcolor}{gray!25}
%\colorlet{tablerowcolor}{gray!12.5}
%\newcommand{\TopRule}{\specialrule{\heavyrulewidth}{0pt}{0pt}}
%\newcommand{\BottomRule}{\specialrule{\lightrulewidth}{0pt}{0pt}}
%\extrarowheight2pt
%\newcolumntype{P}[1]{>{\raggedright}p{#1}}
%\newcolumntype{M}[1]{>{\raggedright}m{#1}}

%%%%%%%%%%%%%%%%%%%%%%%%%%%%%%%%%%%%%%%%%%%%%%%%%%%%%%%%%%%%%%%%%%%%%%%%%%%%%%%%
%% Abbreviations.

\usepackage{xspace}
\newcommand*{\etal}{\emph{et al.}\xspace}
\newcommand*{\etc}{\emph{etc.}\xspace}
\newcommand*{\eg}{\emph{e.g.}\xspace}
\newcommand*{\ie}{\emph{i.e.}\xspace}
\newcommand*{\cf}{\emph{cf.}\xspace}

%%%%%%%%%%%%%%%%%%%%%%%%%%%%%%%%%%%%%%%%%%%%%%%%%%%%%%%%%%%%%%%%%%%%%%%%%%%%%%%%
%% Units.

\usepackage[squaren,Gray]{SIunits}
\addunit{\flops}{flops}

%%%%%%%%%%%%%%%%%%%%%%%%%%%%%%%%%%%%%%%%%%%%%%%%%%%%%%%%%%%%%%%%%%%%%%%%%%%%%%%%
%% Mathematics.

%\useosf % no longer required if osf specified, otherwise after all math
\DeclareSymbolFont{bbold}{U}{bbold}{m}{n}
\DeclareSymbolFontAlphabet{\mathbbold}{bbold}
\RequirePackage{textcomp}  % required for special glyphs

%% Parentheses:
%\def\Paren{}\let\Paren\undefined
\DeclarePairedDelimiterX{\Paren}[1]{(}{)}{#1}
\DeclarePairedDelimiterX{\Brace}[1]{\{}{\}}{#1}
\DeclarePairedDelimiterX{\Brack}[1]{[}{]}{#1}
%\DeclarePairedDelimiterX{\Abs}[1]{\rvert}{\lvert}{#1}
\DeclarePairedDelimiterX{\Abs}[1]{|}{|}{#1}
\DeclarePairedDelimiterX{\Norm}[1]{\lVert}{\rVert}{#1}
\DeclarePairedDelimiterX{\Avg}[1]{\langle}{\rangle}{#1}
\DeclarePairedDelimiterX{\Round}[1]{\lfloor}{\rceil}{#1}
\DeclarePairedDelimiterX{\Floor}[1]{\lfloor}{\rfloor}{#1}
\DeclarePairedDelimiterX{\Ceil}[1]{\lceil}{\rceil}{#1}
\DeclarePairedDelimiterX{\Inner}[1]{\langle}{\rangle}{#1}
\DeclarePairedDelimiterX{\IntRange}[1]{\llbracket}{\rrbracket}{#1}
\DeclarePairedDelimiterX{\Group}[1]{\lgroup}{\rgroup}{#1}
%\DeclarePairedDelimiterX{\FrobNorm}[1]{\lVert}{\rVert_{\Tag{F}}}{#1}
%% A list \List{ELEM}{START}{END}
\DeclarePairedDelimiterXPP{\List}[3]{}{\{}{\}}{_{#2,\ldots,#3}}{#1}

\newcommand*{\delimsize}{}
\newcommand*{\suchthat}{\,\vert\,} % compact version
\newcommand*{\SuchThat}{\,\delimsize\vert\,} % autoscale to surrounding version
\newcommand*{\given}{\,\vert\,} % compact version
\newcommand*{\Given}{\,\delimsize\vert\,} % autoscale to surrounding version
%\newcommand*{\given}{\mathbin{\vert}}
%\newcommand*{\Given}{\mathrel{\delimsize\vert}} % autoscale to surrounding version

%% Upright letters.
\newcommand*{\mathd}{\mathrm{d}}
\newcommand*{\mathe}{\mathrm{e}}
\newcommand*{\mathi}{\mathrm{i}}

%% Operators/functions
\renewcommand*{\Re}{\operatorname{Re}}
\renewcommand*{\Im}{\operatorname{Im}}
\DeclareMathOperator*{\argmin}{arg\,min}
\DeclareMathOperator*{\argmax}{arg\,max}
\DeclareMathOperator{\arc}{arc}
\DeclareMathOperator{\rnd}{rnd}
\DeclareMathOperator{\sign}{sign}
\DeclareMathOperator{\sinc}{sinc}
\DeclareMathOperator{\Card}{Card}
\DeclareMathOperator{\Conv}{conv}
%\DeclareMathOperator{\Det}{Det}
\newcommand*{\Det}{\det}
\DeclareMathOperator{\Diag}{diag}
\DeclareMathOperator{\Dom}{dom}
\DeclareMathOperator{\Prox}{prox}
\DeclareMathOperator{\Rank}{rank}
\DeclareMathOperator{\Sign}{sign}
\DeclareMathOperator{\Span}{span}
\DeclareMathOperator{\Trace}{tr}
\newcommand*{\trace}{\Trace}

%% Statistics.
%\DeclareMathOperator{\Var}{Var}
%\DeclareMathOperator{\Cov}{Cov}
%\DeclareMathOperator{\Expect}{E}
%\newcommand*{\Expect}{\mathbb{E}}
\DeclarePairedDelimiterXPP{\Var}[1]{\mathrm{Var}}(){}{#1}
\DeclarePairedDelimiterXPP{\Cov}[1]{\mathrm{Cov}}(){}{#1}
%\DeclarePairedDelimiterXPP{\Expect}[1]{\mathrm{E}}(){}{#1}
\DeclarePairedDelimiterXPP{\Expect}[1]{\mathbb{E}}(){}{#1}
\DeclarePairedDelimiterXPP{\LogPr}[1]{\ell}(){}{#1}

%% Sets.
\newcommand*{\Set}[1]{\mathbb{#1}}
\newcommand*{\Reals}{\Set{R}}
\newcommand*{\Integers}{\Set{Z}}
\newcommand*{\NaturalNumbers}{\Set{N}}
\newcommand*{\Rationals}{\Set{Q}}
\newcommand*{\Complexes}{\Set{C}}

% For function definitions: $f \from E \to F$.
\newcommand*{\from}{{:}\:}

%% Miscellaneous.
\newcommand*{\Tag}[1]{\mathrm{#1}}
\newcommand*{\PDer}[2]{\frac{\partial#1}{\partial#2}}
\newcommand*{\bydef}{\stackrel{{\scriptscriptstyle \text{def}}}{=}}

%% Fractions.
\newcommand*{\textfrac}[2]{{\textstyle\frac{#1}{#2}}}
\newcommand*{\scriptfrac}[2]{{\scriptstyle\frac{#1}{#2}}}

%% Linear algebra.
\newcommand*{\V}[1]{\boldsymbol{#1}}   % a vector
\newcommand*{\M}[1]{\mathbf{#1}}      % an operator (a matrix)
%\newcommand*{\M}[1]{\boldsymbol{#1}}   % an operator (a matrix)

%% Adjoint/transpose.
\newcommand*{\Adj}{\top}
%\newcommand*{\Adj}{\mathrm{t}}
\newcommand*{\T}{^{\Adj}}
\newcommand*{\mT}{^{-\Adj}}

%% The following definition is to remove/reduce surrounding spacing around ⋅ to
%% denote dot product.
\let\origcdot=\cdot
\renewcommand*{\cdot}{\mathord{\,\mathchar"2201\,}}
%\renewcommand*{\cdot}{\,}
\newcommand*{\Tcdot}{\T\!⋅}

%% Matrices and vectors.
%\newcommand{\ArrayWithDelimiters}[4]{%
%  \left #1\begin{array}{#2}#4\end{array}\right #3}
%\newcommand{\Matrix}[2]{\ArrayWithDelimiters{\lgroup}{#1}{\rgroup}{#2}}
%\newcommand{\Vector}[1]{\ArrayWithDelimiters{\lgroup}{c}{\rgroup}{#1}}
%\newcommand{\TwoByTwoMatrix}[4]{\Matrix{cc}{#1 & #2 \\ #3 & #4 \\}}
%\newcommand{\TwoByTwoSymmetricMatrix}[3]{\TwoByTwoMatrix{#1}{#3}{#3}{#2}}

%\newcommand*{\QuadTerm}[2]{#2\T\cdot#1\cdot#2}
%\newcommand*{\One}{1\hspace*{-0.8ex}1}
%\newcommand*{\Zero}{0\hspace*{-0.8ex}0}
%\newcommand*{\One}{\mathbbm{1}}
%\newcommand*{\Zero}{\mathbbm{0}}
%\newcommand*{\One}{\mathds{1}}
%\newcommand*{\Zero}{\mathds{O}}
%\newcommand*{\Zero}{\V 0}
%\newcommand*{\One}{\V 1}

%------------------------------------------------------------------------------
% Ref: Alexander R. Perlis., "A complement to \smash, \llap, and \rlap,"
%      TUGboat, Vol. 22, pp. 350-352, 2001.
%      <http://math.arizona.edu/~aprl/publications/mathclap/>
%
% For comparison, the existing overlap macros:
% \def\llap#1{\hbox to 0pt{\hss#1}}
% \def\rlap#1{\hbox to 0pt{#1\hss}}
  \def\clap#1{\hbox to 0pt{\hss#1\hss}}
  \def\mathllap{\mathpalette\mathllapinternal}
  \def\mathrlap{\mathpalette\mathrlapinternal}
  \def\mathclap{\mathpalette\mathclapinternal}
  \def\mathllapinternal#1#2{\llap{$\mathsurround=0pt#1{#2}$}}
  \def\mathrlapinternal#1#2{\rlap{$\mathsurround=0pt#1{#2}$}}
  \def\mathclapinternal#1#2{\clap{$\mathsurround=0pt#1{#2}$}}
%------------------------------------------------------------------------------
